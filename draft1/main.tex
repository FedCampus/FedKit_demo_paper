%File: anonymous-submission-latex-2024.tex
\documentclass[letterpaper]{article} % DO NOT CHANGE THIS
\usepackage[submission]{aaai24}  % DO NOT CHANGE THIS
\usepackage{times}  % DO NOT CHANGE THIS
\usepackage{helvet}  % DO NOT CHANGE THIS
\usepackage{courier}  % DO NOT CHANGE THIS
\usepackage[hyphens]{url}  % DO NOT CHANGE THIS
\usepackage{graphicx} % DO NOT CHANGE THIS
\urlstyle{rm} % DO NOT CHANGE THIS
\def\UrlFont{\rm}  % DO NOT CHANGE THIS
\usepackage{natbib}  % DO NOT CHANGE THIS AND DO NOT ADD ANY OPTIONS TO IT
\usepackage{caption} % DO NOT CHANGE THIS AND DO NOT ADD ANY OPTIONS TO IT
\frenchspacing  % DO NOT CHANGE THIS
\setlength{\pdfpagewidth}{8.5in} % DO NOT CHANGE THIS
\setlength{\pdfpageheight}{11in} % DO NOT CHANGE THIS
%
% These are recommended to typeset algorithms but not required. See the subsubsection on algorithms. Remove them if you don't have algorithms in your paper.
\usepackage{algorithm}
\usepackage{algorithmic}

% Custom packages
\usepackage{tikz}
\usepackage[inline]{enumitem}

%
% These are are recommended to typeset listings but not required. See the subsubsection on listing. Remove this block if you don't have listings in your paper.
\usepackage{newfloat}
\usepackage{listings}
\DeclareCaptionStyle{ruled}{labelfont=normalfont,labelsep=colon,strut=off} % DO NOT CHANGE THIS
\lstset{%
	basicstyle={\footnotesize\ttfamily},% footnotesize acceptable for monospace
	numbers=left,numberstyle=\footnotesize,xleftmargin=2em,% show line numbers, remove this entire line if you don't want the numbers.
	aboveskip=0pt,belowskip=0pt,%
	showstringspaces=false,tabsize=2,breaklines=true}
\floatstyle{ruled}
\newfloat{listing}{tb}{lst}{}
\floatname{listing}{Listing}
%
% Keep the \pdfinfo as shown here. There's no need
% for you to add the /Title and /Author tags.
\pdfinfo{
/TemplateVersion (2024.1)
}

% DISALLOWED PACKAGES
% \usepackage{authblk} -- This package is specifically forbidden
% \usepackage{balance} -- This package is specifically forbidden
% \usepackage{color (if used in text)
% \usepackage{CJK} -- This package is specifically forbidden
% \usepackage{float} -- This package is specifically forbidden
% \usepackage{flushend} -- This package is specifically forbidden
% \usepackage{fontenc} -- This package is specifically forbidden
% \usepackage{fullpage} -- This package is specifically forbidden
% \usepackage{geometry} -- This package is specifically forbidden
% \usepackage{grffile} -- This package is specifically forbidden
% \usepackage{hyperref} -- This package is specifically forbidden
% \usepackage{navigator} -- This package is specifically forbidden
% (or any other package that embeds links such as navigator or hyperref)
% \indentfirst} -- This package is specifically forbidden
% \layout} -- This package is specifically forbidden
% \multicol} -- This package is specifically forbidden
% \nameref} -- This package is specifically forbidden
% \usepackage{savetrees} -- This package is specifically forbidden
% \usepackage{setspace} -- This package is specifically forbidden
% \usepackage{stfloats} -- This package is specifically forbidden
% \usepackage{tabu} -- This package is specifically forbidden
% \usepackage{titlesec} -- This package is specifically forbidden
% \usepackage{tocbibind} -- This package is specifically forbidden
% \usepackage{ulem} -- This package is specifically forbidden
% \usepackage{wrapfig} -- This package is specifically forbidden
% DISALLOWED COMMANDS
% \nocopyright -- Your paper will not be published if you use this command
% \addtolength -- This command may not be used
% \balance -- This command may not be used
% \baselinestretch -- Your paper will not be published if you use this command
% \clearpage -- No page breaks of any kind may be used for the final version of your paper
% \columnsep -- This command may not be used
% \newpage -- No page breaks of any kind may be used for the final version of your paper
% \pagebreak -- No page breaks of any kind may be used for the final version of your paperr
% \pagestyle -- This command may not be used
% \tiny -- This is not an acceptable font size.
% \vspace{- -- No negative value may be used in proximity of a caption, figure, table, section, subsection, subsubsection, or reference
% \vskip{- -- No negative value may be used to alter spacing above or below a caption, figure, table, section, subsection, subsubsection, or reference

\setcounter{secnumdepth}{0} %May be changed to 1 or 2 if section numbers are desired.

% The file aaai24.sty is the style file for AAAI Press
% proceedings, working notes, and technical reports.
%

% Title

% Your title must be in mixed case, not sentence case.
% That means all verbs (including short verbs like be, is, using,and go),
% nouns, adverbs, adjectives should be capitalized, including both words in hyphenated terms, while
% articles, conjunctions, and prepositions are lower case unless they
% directly follow a colon or long dash
\title{FedKit---Cross-Platform On-Smartphone Federated Learning}
\author{}
\affiliations{}

\begin{document}

\maketitle

\begin{abstract}
    TODO
\end{abstract}

\section{Introduction}

% TODO: Come back after finishing Conclusion.

\section{System Description}

Our FL setting is a distributed system.
The Backend orchestrates the FL process and manages data.
Clients, in the form of smartphone apps,
connect to the Backend to prepare and run FL tasks.
We propose two major innovations:
\begin{enumerate*}[label=\arabic*)]
    \item cross-smartphone-platform FL support by converting ML models
    into ones that Android and iOS natively support, and
    \item flexible FL customization in production by adding a preparation stage
    before FL training
\end{enumerate*}.
We elaborate on these innovations next.

\subsection{Cross-Smartphone-Platform FL}
We support federating ML models training on smartphones
with different operating systems.
For the same ML model developed in Python on a desktop computer,
we can federate training it both on Android and iOS smartphones,
and we aggregate parameters across these platforms.
This is significant because it
\begin{enumerate*}[label=\arabic*)]
    \item makes it feasible for academic researchers to
        conduct real-world cross-platform FL research, and
    \item allows companies to use the data on smartphones to train their FL
        models
\end{enumerate*}.
Cross-smartphone-platform FL is difficult because it means we need to interface
Python-incompatible environments with different operating systems and
ML frameworks.
This is the major reason why other FL frameworks do not have
cross-smartphone-platform support.
% TODO: Chart of FL frameworks' platform support.

We achieve cross-smartphone-platform FL by bridging between
desktop ML and native smartphone ML in three steps:
\begin{enumerate*}[label=\arabic*.]
    \item convert the ML models from desktop ML frameworks (e.g. TensorFlow)
        into formats natively supported by each smartphone platform,
    \item build an unified interface to run the converted models on smartphones
        using native ML frameworks, and
    \item transmit model parameters in an unified format for
        cross-platform aggregation
\end{enumerate*}.
% TODO: Figure of ML model conversion.
Using the native frameworks is important because
we can then leverage the GPUs and NPUs on these resource-constraint smartphones.
These frameworks are Google's TensorFlow Lite (TFLite) on Android and
Apple's Core ML on iOS,
and they impose difficulties at different stages.
TFLite model definitions are excessively open-ended,
so we standardize four methods to read and update parameters,
train and infer, and
provide a Python converter to generate standardized TFLite models from
any TensorFlow models.
Running TFLite on smartphones are straightforward interpreter method calls.
In contrast, Core ML models have defined forms and an official Python converter,
but on-device training is artificially restricted.
To work around the restrictions,
we manually read and update Core ML models' underlying ProtoBuf representations
to update their parameters and configure the training data.
In the end, we have unified native training APIs both on Android and iOS,
which, more importantly, yield model parameters in the same representation.
On both platforms, neural network's parameters are represented as byte arrays
for each layer.
We align these layers in the same order on both platforms,
so when the Backend receives them,
it can aggregate them without any knowledge of
the implementation of the clients.
This enables us to aggregate parameters from different platforms.
Cross-smartphone-platform aggregation,
together with cross-smartphone-platform training and model conversion,
allows true cross-smartphone-platform FL.

\subsection{Flexible In-Production FL Customization}
\newcommand{\model}{$M$}
\newcommand{\fs}{$S_\mathrm F$}
We provide the flexibility to customize ML models and FL algorithms
in production.
This is crucial because, to iterate on ML models and explore new FL algorithms,
researchers must have the ability to change them.
However, customization in production is difficult because
the clients in on-smartphone FL are deployed as smartphone apps and
remain largely static until an update comes.
App updates make fast research iteration infeasible because of
app stores' slow review processes and
the wait for the users to apply the update.
Instead, we add a preparation stage in our FL workflow and
utilizes the flexibility of our Backend to
maximize the potential of FL customization.

Our FL workflow provides extra flexibility in FL customization in production by
\begin{enumerate*}[label=\arabic*)]
    \item moving as many responsibilities to the Backend as possible and
    \item separating out the FL responsibility
\end{enumerate*}.
We first move to the Backend the responsibility to
decide which ML model to train.
% TODO: Figure of structure with workflow, numbered.
As shown in Fig. , % TODO: ref
clients start by requesting the Backend for an ML model \model{} that fits
the kind of data they have.
They then initiate the training by requesting an FL Server \fs{} for \model{}
from the Backend and federate learning by communicating with \fs{}
(Fig. ). % TODO: ref
This is where we move to the Backend the responsibility to track FL sessions
and separate out the FL responsibility to FL Servers.
The Backend spawns FL Servers as sub-processes and track them to decide whether
to spawn new FL Servers or let new clients join existing ones.
The FL Servers manages the actual FL training and evaluation process.
Both the Backend and the FL Servers are easy to customize in production because
they are Python programs using, respectively,
the Django REST Framework and the Flower FL Framework.
By simply uploading new ML models to the associated data types and
changing the FL algorithm code for the FL Servers in Python,
researchers can flexibly customize the FL process in production.


\section{Conclusion and Future Work}


\appendix

\section*{Acknowledgments}
TODO

\bigskip

\bibliography{main}

\end{document}
