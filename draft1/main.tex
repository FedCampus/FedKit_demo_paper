%File: anonymous-submission-latex-2024.tex
\documentclass[letterpaper]{article} % DO NOT CHANGE THIS
\usepackage[submission]{aaai24}  % DO NOT CHANGE THIS
\usepackage{times}  % DO NOT CHANGE THIS
\usepackage{helvet}  % DO NOT CHANGE THIS
\usepackage{courier}  % DO NOT CHANGE THIS
\usepackage[hyphens]{url}  % DO NOT CHANGE THIS
\usepackage{graphicx} % DO NOT CHANGE THIS
\urlstyle{rm} % DO NOT CHANGE THIS
\def\UrlFont{\rm}  % DO NOT CHANGE THIS
\usepackage{natbib}  % DO NOT CHANGE THIS AND DO NOT ADD ANY OPTIONS TO IT
\usepackage{caption} % DO NOT CHANGE THIS AND DO NOT ADD ANY OPTIONS TO IT
\frenchspacing  % DO NOT CHANGE THIS
\setlength{\pdfpagewidth}{8.5in} % DO NOT CHANGE THIS
\setlength{\pdfpageheight}{11in} % DO NOT CHANGE THIS
%
% These are recommended to typeset algorithms but not required. See the subsubsection on algorithms. Remove them if you don't have algorithms in your paper.
\usepackage{algorithm}
\usepackage{algorithmic}

% Custom packages
\usepackage{tikz}
\usepackage[inline]{enumitem}

%
% These are are recommended to typeset listings but not required. See the subsubsection on listing. Remove this block if you don't have listings in your paper.
\usepackage{newfloat}
\usepackage{listings}
\DeclareCaptionStyle{ruled}{labelfont=normalfont,labelsep=colon,strut=off} % DO NOT CHANGE THIS
\lstset{%
	basicstyle={\footnotesize\ttfamily},% footnotesize acceptable for monospace
	numbers=left,numberstyle=\footnotesize,xleftmargin=2em,% show line numbers, remove this entire line if you don't want the numbers.
	aboveskip=0pt,belowskip=0pt,%
	showstringspaces=false,tabsize=2,breaklines=true}
\floatstyle{ruled}
\newfloat{listing}{tb}{lst}{}
\floatname{listing}{Listing}
%
% Keep the \pdfinfo as shown here. There's no need
% for you to add the /Title and /Author tags.
\pdfinfo{
/TemplateVersion (2024.1)
}

% DISALLOWED PACKAGES
% \usepackage{authblk} -- This package is specifically forbidden
% \usepackage{balance} -- This package is specifically forbidden
% \usepackage{color (if used in text)
% \usepackage{CJK} -- This package is specifically forbidden
% \usepackage{float} -- This package is specifically forbidden
% \usepackage{flushend} -- This package is specifically forbidden
% \usepackage{fontenc} -- This package is specifically forbidden
% \usepackage{fullpage} -- This package is specifically forbidden
% \usepackage{geometry} -- This package is specifically forbidden
% \usepackage{grffile} -- This package is specifically forbidden
% \usepackage{hyperref} -- This package is specifically forbidden
% \usepackage{navigator} -- This package is specifically forbidden
% (or any other package that embeds links such as navigator or hyperref)
% \indentfirst} -- This package is specifically forbidden
% \layout} -- This package is specifically forbidden
% \multicol} -- This package is specifically forbidden
% \nameref} -- This package is specifically forbidden
% \usepackage{savetrees} -- This package is specifically forbidden
% \usepackage{setspace} -- This package is specifically forbidden
% \usepackage{stfloats} -- This package is specifically forbidden
% \usepackage{tabu} -- This package is specifically forbidden
% \usepackage{titlesec} -- This package is specifically forbidden
% \usepackage{tocbibind} -- This package is specifically forbidden
% \usepackage{ulem} -- This package is specifically forbidden
% \usepackage{wrapfig} -- This package is specifically forbidden
% DISALLOWED COMMANDS
% \nocopyright -- Your paper will not be published if you use this command
% \addtolength -- This command may not be used
% \balance -- This command may not be used
% \baselinestretch -- Your paper will not be published if you use this command
% \clearpage -- No page breaks of any kind may be used for the final version of your paper
% \columnsep -- This command may not be used
% \newpage -- No page breaks of any kind may be used for the final version of your paper
% \pagebreak -- No page breaks of any kind may be used for the final version of your paperr
% \pagestyle -- This command may not be used
% \tiny -- This is not an acceptable font size.
% \vspace{- -- No negative value may be used in proximity of a caption, figure, table, section, subsection, subsubsection, or reference
% \vskip{- -- No negative value may be used to alter spacing above or below a caption, figure, table, section, subsection, subsubsection, or reference

\setcounter{secnumdepth}{0} %May be changed to 1 or 2 if section numbers are desired.

% The file aaai24.sty is the style file for AAAI Press
% proceedings, working notes, and technical reports.
%

% Title

% Your title must be in mixed case, not sentence case.
% That means all verbs (including short verbs like be, is, using,and go),
% nouns, adverbs, adjectives should be capitalized, including both words in hyphenated terms, while
% articles, conjunctions, and prepositions are lower case unless they
% directly follow a colon or long dash
\title{FedKit---Cross-Platform On-Smartphone Federated Learning}
\author{}
\affiliations{}

\begin{document}

\maketitle

\begin{abstract}
    TODO
\end{abstract}

\section{Introduction}

% TODO: Come back after finishing Conclusion.

TODO

FedKit presents the following key highlights:

\begin{enumerate*}[label=\arabic*)]
    \item Academic Research: FedKit enables FL academic research to test FL algorithms using real-world data from diverse smartphone platforms, providing valuable insights under real-world circumstances.
    \item Industry Application: FedKit allows the industry to harness the vast amount of data available on smartphones for FL applications, offering potential improvements in various domains.
\end{enumerate*}.

% TODO: Chart of FL frameworks' platform support.

\section{System Description}

FedKit is a sophisticated client-server FL system that leverages a robust Backend as the server, complemented by \textit{cross-platform smartphones} as clients.
% TODO: Figure of system structure.
The clients perform local training on private on-mobile data and collaboratively contribute to the training of the global model under the Backend's supervision.
Despite the system's apparent simplicity and elegance, implementing a mobile-based FL system introduces specific challenges due to the intricacies of cross-platform mobile development.
To address these challenges, we propose FedKit to support cross-platform on-smartphone FL that emphasizes two key aspects that distinguish it from existing solutions:
\begin{enumerate*}[label=\arabic*)]
    \item cross-smartphone-platform FL support using
        native smartphone ML frameworks, and
    \item flexible FL customization in production enabled by
        our custom FL workflow
\end{enumerate*}.

\subsection{Cross-Smartphone-Platform FL}

FedKit introduces a novel approach to FL on smartphones
with different operating systems.
It simplifies on-smartphone FL for researchers by empowering them to
use Python to
\begin{enumerate*}[label=\arabic*)]
    \item create ML models to train on different smartphone platforms at
        native performance, and
    \item schedule FL and aggregate models across
        heterogeneous smartphone platforms
\end{enumerate*}.
The primary challenges that block existing solutions from
achieving cross-smartphone-platform FL is
\begin{enumerate*}[label=\arabic*)]
    \item the foreignness in cross-platform mobile development and
    \item the complexity in on-smartphone ML training
\end{enumerate*}.
To overcome the above challenges, we propose a three-step approach:
\begin{enumerate*}[label=\arabic*.]
    \item Convert the ML models from desktop ML frameworks (e.g., TensorFlow)
        into formats natively supported by each smartphone platform,
    \item Build a unified interface to run the converted models on smartphones
        using native ML frameworks, and
    \item Aggregate model parameters across platforms using a unified format
\end{enumerate*}.

% TODO: Figure of ML model conversion.
FedKit's implementation does the heavy-lifting by leveraging
native mobile ML frameworks, specifically,
Google's TensorFlow Lite (TFLite) on Android and Apple's Core ML on iOS.
These frameworks are chosen for their superior utilization of GPUs and NPUs,
which are essential for accelerating training on resource-constrained smartphones.
The drawback is that they impose unique and intricate challenges,
so we had to employ exotic solutions to handle them.
\begin{enumerate*}[label=\arabic*)]
    \item TFLite Model Standardization:
        A TFLite model is a collection of user-defined methods to be run on
        its interpreter.
        To provide a unified training interface and simplify model construction,
        we standardize four methods
        (read and update parameters, train, and infer),
        and utilizing them to implement FL on Android.
        Additionally, to free researchers from manually implementing and
        converting TFLite models,
        we provide a Python converter that supports any TensorFlow model.
    \item Core ML Jail-Break:
        Core ML models have defined forms and an official Python converter.
        However, in our experiments, we confirmed that
        it disallows setting model parameters,
        which makes FL impossible.
        To overcome this restriction, we manually alter
        Core ML models' underlying ProtoBuf representations to
        update their parameters and configure the training data.
\end{enumerate*}.

Furthermore, to aggregate local models from different platforms,
we carefully represent model parameters in a unified format by
sequentially retrieving each layer's parameters as a byte array.
Consequently,
FedKit not only results in unified native training APIs on
both Android and iOS, but also
allow for seamless cross-platform aggregation.

\subsection{Flexible In-Production FL Customization}
\newcommand{\model}{$M$}
\newcommand{\fs}{$S_\mathrm F$}
FedKit provides the flexibility to customize ML models and FL algorithms
in production.
It gives FL researchers the crucial ability to
conveniently iterate on ML models and explore new FL algorithms.
To allow these changes, we identify the primary challenge when
customizing existing FL solutions in production:
with the ML models and the training procedures hard-coded in the client app,
modifying them requires a complete app update,
which is largely governed by app stores and app users.
Instead, to maintain control in researchers' hands,
we propose a three-step FL workflow from clients' perspective:
\begin{enumerate*}[label=\arabic*.]
    \item Model Request: request the Backend for an ML model \model{} that fits
        their platform and training data type.
    \item Training Initiation: request the Backend for an FL Server \fs{}
        for \model.
    \item Training: regular FL training by communicating with \fs.
\end{enumerate*}
% TODO: Figure of structure with workflow, numbered.

FedKit's FL workflow implementation provides extra flexibility in
FL customization in production by
shifting as many responsibilities to the Backend as possible.
To support deploying new ML models from the Backend,
we rely on the Backend to decide which model \model{} to train in Model Request.
To control each FL session in isolation, in Training Initiation,
we utilize the Backend to either
\begin{enumerate*}[label=\arabic*)]
    \item assign new clients to existing FL Servers or
    \item spawn new ones as Python subprocesses
\end{enumerate*}.
Such isolation enables federating learning multiple models simultaneously, and
allows the users to customize the FL Server for each model.
Moreover, customizing individual FL Servers is well-supported because
they utilize the user-friendly Flower FL Framework to
allows for custom FL algorithms to schedule training and evaluation.
With our Django Backend and Flower FL Servers,
we offer FL researchers strong control and flexibility to customize FL
using Python in production.


\section{Conclusion and Future Work}


\appendix

\section*{Acknowledgments}
TODO

\bigskip

\bibliography{main}

\end{document}
