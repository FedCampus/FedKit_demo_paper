\documentclass[conference]{IEEEtran}
\IEEEoverridecommandlockouts
% The preceding line is only needed to identify funding in the first footnote. If that is unneeded, please comment it out.
\usepackage{cite}
\usepackage{amsmath,amssymb,amsfonts}
\usepackage{algorithmic}
\usepackage{graphicx}
\usepackage{textcomp}
\usepackage{xcolor}
\usepackage[hyphens]{url}
\usepackage{tikz}
\usepackage[inline]{enumitem}
\usepackage{pifont} 
\usepackage{listings}
\graphicspath{ {./images/} }
\def\BibTeX{{\rm B\kern-.05em{\sc i\kern-.025em b}\kern-.08em
    T\kern-.1667em\lower.7ex\hbox{E}\kern-.125emX}}
\begin{document}
\bstctlcite{IEEEexample:BSTcontrol}

\newcommand*\circled[1]{\tikz[baseline=(char.base)]{
        \node[shape=circle,draw,inner sep=.6pt] (char) {#1};}}
\newcommand{\FedCampus}{\textsc{FedCampus}}
\newcommand{\FedKit}{\textsc{FedKit}}
\newcommand{\challa}{\textbf{(a)}}
\newcommand{\challb}{\textbf{(b)}}
\newcommand{\challc}{\textbf{(c)}}

\title
{\FedKit{}: Enabling Cross-Platform Federated Learning for Android and iOS}

\author{\IEEEauthorblockN{Sichang He\IEEEauthorrefmark{1}, Beilong Tang\IEEEauthorrefmark{1}, Boyan Zhang\IEEEauthorrefmark{1}, Jiaoqi Shao\IEEEauthorrefmark{1}\IEEEauthorrefmark{2}, Xiaomin Ouyang\IEEEauthorrefmark{3}, Daniel\,Nata Nugraha\IEEEauthorrefmark{4}, Bing Luo\IEEEauthorrefmark{1}}
    \IEEEauthorblockA{\IEEEauthorrefmark{1}% Data Science Research Center,
        Duke Kunshan University, Jiangsu, China,
    % }
    % \IEEEauthorblockA{
        \IEEEauthorrefmark{2}% Department of Electronic and Computer Engineering,
        The Hong Kong University of Science and Technology, China,\\
    % }
    % \IEEEauthorblockA{
        \IEEEauthorrefmark{3}% Department of Electrical and Computer Engineering,
        University of California, Los Angeles, USA,
    % }
    % \IEEEauthorblockA{
        \IEEEauthorrefmark{4}Flower Labs GmbH, Winterhuder Weg 29, 22085 Hamburg, Germany
    }% <-this % stops an unwanted space
}

\maketitle

\begin{abstract}
We present \FedKit{}, a federated learning (FL) system tailored for
cross-platform FL research on \textit{Android and iOS} devices.
\FedKit{} pipelines cross-platform FL development by
enabling model conversion,
hardware-accelerated training,
and cross-platform model aggregation.
Our FL workflow supports flexible machine learning operations (MLOps) in production,
facilitating continuous model delivery and training.
We have deployed \FedKit{} in a real-world use case for
health data analysis on Duke Kunshan University (DKU) campus,
demonstrating its effectiveness.
% TODO: Xiaomin: in xxx various platforms such as xxx
\FedKit{} is open-source at \url{https://github.com/FedCampus/FedKit}.
\end{abstract}

% \begin{IEEEkeywords}
% \end{IEEEkeywords}

\section{Motivation and Analysis}

FL is a promising technique to collaboratively train shared ML models on
end devices while retaining private local data~\cite{wang2023federated}.
However, most FL research experiments with simulations on desktop computers,
which may overlook constraints in realistic FL applications.
To enhance FL algorithm design with real-world data,
we aim to build a practical mobile FL system for
harnessing real user data.

However,
existing accessible mobile FL systems exhibit important limitations,
as we outline in Table~\ref{tbl:fn-systems}.
Specifically, we identify three key challenges:
\challa~to collaboratively train the same models across
our users' diverse smartphones,
we require \textbf{cross-platform on-device training and model aggregation};
\challb~to facilitate model development,
we require a \textbf{friendly development environment for data scientists};
% TODO: xiaomin: The key requirement for "friendly" is not clear
\challc~to customize FL algorithms and update models in production,
we require a \textbf{customizable and streamlined deployment procedure}.
% TODO: xiaomin: Again, avoid using these in problem description, we should clarify what customized and streamlined mean.

\section{Proposed Solution}

\FedKit{} is designed to enable practical \textbf{cross-platform} FL research on
\textbf{Android and iOS}.
In Section~\ref{sec:pipeline},
we present an FL pipeline to convert Python-based models,
and train and aggregate them across platforms,
addressing challenge~\challa{}.
For challenge~\challc{},
our FL workflow enables \textbf{flexible MLOps} from the backend in production,
as we detail in Section~\ref{sec:mlops}.
Both the FL pipeline and workflow provide a familiar Python environment,
addressing challenge~\challb{}.
Overall,
\FedKit{} facilitates FL across Android and iOS client devices,
coordinated by a single Backend server.
Each client trains a local model with private data,
and the Backend performs cross-platform aggregation of these local models to update the global model.

\subsection{Cross-Platform FL Model Pipeline}
\label{sec:pipeline}

To enable cross-platform FL,
especially \textit{cross-platform aggregation},
we propose a pipeline comprising
\textit{model conversion} and
\textit{unified training APIs},
as shown in Fig.~\ref{cross_fl}.

\begin{table}
    \centering
    \small
    % \setlength{\tabcolsep}{2pt}
    \newcommand{\Ys}{\ding{51}}
    \newcommand{\No}{\ding{55}}
\begin{tabular}{lcccccc}
Functionality               & \cite{he2020fedml}
                                    & \cite{madrigal2023project}
                                            & \cite{mathur2021ondevice}
                                                    & \cite{hall2021syft}
                                                            & \FedKit{}\\
\hline
Android-Only                & \Ys   & \Ys   & \Ys   & \Ys   & \Ys   \\
iOS-Only                    & \No   & \No   & \Ys   & \Ys   & \Ys   \\
Cross-Platform Aggregation  & \No   & \No   & \No   & \Ys   & \Ys   \\\hline
Training Acceleration       & \Ys   & \Ys   & \Ys   & \No   & \Ys   \\
MLOps                       & \Ys   & \Ys   & \No   & \No   & \Ys   \\
Open-Source Backend         & \No   & \No   & \Ys   & \Ys   & \Ys   \\
\end{tabular}
\caption{Functionality Comparison among On-Smartphone FL Systems.}
\label{tbl:fn-systems}
\end{table}

\begin{figure}
    \centering
    \includegraphics*[width=\linewidth]{model_pipeline.pdf}
    \caption{\FedKit{} Model Pipeline for Cross-Platform FL.}
    \label{cross_fl}
\end{figure}

\subsubsection{Model Conversion}
% TODO: Jiaqi: State: why model conversion and what are the implementation challenges 
We begin in Python by converting models into formats compatible with
Android (TensorFlow Lite or TFLite) and iOS (Core ML).
Core ML defines a fixed model structure and provides
the official converter CoreMLTools.
For TFLite, we \textit{standardized} a model format, and then
developed a compliant TensorFlow converter.
This standardized format includes
four essential FL methods
(\textsf{train}, \textsf{infer}, \textsf{parameters},
and \textsf{restore}).


\subsubsection{Unified Training APIs}
\FedKit{} provides \textit{TFLite Trainer} and \textit{Core ML Trainer} to
train the converted models on Android and iOS devices,
utilizing GPU and NPU acceleration.
Moreover, both trainers expose unified APIs for
\textit{retrieving and setting model parameters,
    model fitting, and evaluation}.
On Android, these APIs invoke the TFLite interpreter to call
our \textit{standardized methods} defined in \textit{Model Conversion}.
However, on iOS, our experimentation revealed that
Core ML \textit{forbids} directly setting parameters, which could impede FL.
% TODO: Jiaqi: I think the challenge should be highlight at the beginning of this paragraph. 
To overcome this constraint,
we modify the underlying ProtoBuf representations of
Core ML models.
Specifically,
we employ Swift code generated from the relevant ProtoBuf definition files,
and navigate nested model definitions to access parameters on iOS.
% TODO: xiaomin: should be expanded
Consequently, our unified training APIs exhibit comparable functionality on
both iOS and Android platforms.


\subsubsection{Cross-Platform Aggregation}
Aggregation necessitates
\textit{uniform parameter representations},
which poses primary challenges in
retrieving and setting parameters for TFLite and Core ML.
1)
The TFLite interpreter only accepts inputs/outputs as maps from \textit{names} to
\textit{tensors}.
% TODO: Jiaqi: should followed by (2).  And then, provide the details of your solution
Therefore, during \textit{Model Conversion},
we assign index-based names to each parameter layer and
dynamically generate the concrete methods that accept these arguments.
During training, we call the methods with these index-based names to
access parameters.
2)
Core ML permits \textit{only} specific layers to be \textit{updatable} and
only provides their parameters post-training.
Thus, to obtain \textit{other layers'} parameters,
we implemented a solution using ProtoBuf manipulation.
This approach involves recording layer information
during \textit{Model Conversion} and
utilizing it during training.
Finally, these unified parameters enable seamless cross-platform aggregation.

\subsection{Flexible MLOps in Production}
% TODO: xiaomin: What are the challenges of MLOPs in Production?
\label{sec:mlops}
\newcommand{\model}{$M$}
\newcommand{\fs}{$S_\mathrm F$}
\FedKit{} empowers researchers to deploy models and algorithms continuously (MLOps)
in production.
Leveraging our complete control of the self-hosted Backend,
\FedKit{}'s three-step FL workflow
facilitates continuous delivery and training,
as illustrated in Fig.~\ref{fig:fl-workflow}.

\begin{figure}
    \centering
    \includegraphics*[width=0.9\linewidth]{fl_workflow.pdf}
    \caption{\FedKit{} FL Workflow.}
    \label{fig:fl-workflow}
\end{figure}

\subsubsection{Continuous Cross-Platform Model Delivery}
Traditionally, models for on-device training are \textit{embedded} into client apps.
However, this approach couples model delivery with app updates,
resulting in complexities when submitting apps to app stores and garnering user adoption.

Circumventing this complexity,
\FedKit{} enables \textit{continuous model delivery without app updates} by
decoupling models from clients through \textit{Model Request},
which allows new model deployment by uploading to the Backend.
Specifically, clients request the Backend for a model (TFLite/Core ML)
aligned with
their platform (Android/iOS) and training data type.
The Backend selects the appropriate model \model{} from the database and
responds with detailed model information.
Consequently, \textit{Model Request} delivers the TFLite or Core ML model
to clients for FL training.

\subsubsection{Customizable Continuous FL Training}
\FedKit{} manages continuous FL training by allowing \textit{multiple parallel FL training sessions}
through \textit{FL Server Setup}.
When clients request an FL Server to
train their chosen model \model{},
the Backend either reuses a suitable FL Server \fs{} if it exists,
or spawns a new one.
Each FL Server
operates as an independent Python subprocess of the Django Backend,
occupying its own port that
clients connect to for FL Training.
This dynamic approach ensures that
newly delivered models can be immediately trained with new FL Servers
without affecting ongoing ones.
Furthermore,
these FL Servers employ the Flower FL Framework for
scheduling training and evaluation.
This decision empowers our FL Servers to leverage Flower's flexibility and
allow for FL algorithm customizations in Python.

\section{Live Demonstrations}

We demonstrate \FedKit{}'s effectiveness in two settings.\footnote{
    Demo video: \url{https://www.youtube.com/watch?v=TONTBkp_l6M}.
}

\subsubsection{Model Deployment on Demo Android/iOS App}
We demonstrate FL among
devices running an example Flutter client app,
and a laptop running a \FedKit{} Backend.
First, we demonstrate our seamless \textit{FL model pipeline}.
We convert a TensorFlow MNIST model and
conduct FL with it across an Android and an iOS device.
Second, to showcase \textit{MLOps},
we modify the model and deploy its new version.
As outlined in Table~\ref{tbl:demo-stats},
our telemetry shows that
the iOS device is over 5$\times$ faster in local training despite
having 0.5$\times$ RAM,
illustrating how \FedKit{} will provide real-world statistics to
enhance FL algorithm design.

\begin{table}
    \centering
    \small
    \setlength{\tabcolsep}{5pt}
\begin{tabular}{llllr}
Device      & SoC               & Acceleration  & RAM   & Time   \\\hline
Nova 9 Pro  & Snapdragon 778G   & GPU           & 8GB   & 3.583s \\
iPhone 13   & A15 Bionic        & GPU/NPU       & 4GB   & 0.656s \\
\end{tabular}
\caption{Configurations of Devices and Average Local Training Time Per Round
    (Two Local Epochs) in A Previous Demo Run.
}
\label{tbl:demo-stats}
\end{table}

\subsubsection{\FedCampus{}}
% TODO: Introduce FedCampus.
The \FedCampus{} Android/iOS app leverages health data from
DKU participants' smartwatches to perform FL on
a sleep-efficiency prediction model.
As illustrated in Fig.~\ref{fig:fedcampus},
we showcase our self-hosted Backend,
and display the real-time logs and losses on a connected laptop.
The results showcased a significant reduction in the model's training loss,
% TODO: xiaomin: and xxx (something about continous model training across platforms)
demonstrating \FedKit{}'s effectiveness in real-world scenarios.

\begin{figure}
    \centering
    \includegraphics*[width=0.9\linewidth]{fedcampus.pdf}
    \caption{\FedCampus{} Experiment Setup on DKU Campus.}
    \label{fig:fedcampus}
\end{figure}

\bibliographystyle{IEEEtran}
\bibliography{main}

\end{document}
